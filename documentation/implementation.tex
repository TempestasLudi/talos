\documentclass{article}

\usepackage{enumerate}
\usepackage{datetime2}

\title{The contents of a Talos implementation}

\begin{document}
	\maketitle

	\section{Types of implementations}
	Different programming languages express different ideas about what a program should do and how it should work. Therefore, it is only logical that in some cases these specs will not be implemented perfectly. For example, in functional programming languages, it might be necessary for functions to return a modified object. The aim of this specification is only to achieve some level of consistency in the function of the different implementations. Whenever you diverge in an implementation, add a comment about that to its documentation.
	
	\section{The core}
	The core of the system is determining who has access to what. To that end, suppose the permission rules have already been loaded. Then, a version of the following method should be available:\\
	\texttt{hasAccess(role, path, variables, sets)}\\
	This function calculates whether the specified role has access to the resource indicated by \texttt{path}, given the specified variables and sets. It returns a boolean. The types of the arguments are:\\
	\begin{tabular}{l|p{.7\textwidth}}
		\textbf{Name} & \textbf{Type}\\\hline
		\texttt{role} & string\\\hline
		\texttt{path} & string\\\hline
		\texttt{variables} & set of key-value pairs, with both keys and values being strings\\\hline
		\texttt{sets} & set of key-value pairs, the keys being strings, the values sets of strings
	\end{tabular}
	
	\section{IO}
	\subsection{Basic functionality}
	The system should at least support some version of the following three methods:
	\begin{enumerate}[1.]
		\item
			\texttt{loadPermissions(permissions)}\\
			This function loads the permission and inheritance rules specified in \texttt{permissions}. It can, if deemed useful by the implementer, construct the permission tree for every role, although this can also be done when checking access for a specific role. The function does not return a value. \texttt{permissions} is a string.
		\item \texttt{loadPermissionsFile(filename)}\\
			This function opens the file \texttt{filename} and loads the permission and inheritance rules specified therein. It can, if deemed useful by the implementer, construct the permission tree for every role, although this can also be done when checking access for a specific role. The function does not return a value. \texttt{filename} is a string.
		\item \texttt{clearPermissions()}\\
			This function unloads the permissions that have previously been loaded.
	\end{enumerate}

	\subsection{Optional functionality}
	Usually when systems grow, more and more data is moved to a database. The permission and inheritance rules could be put in a database table. A version of the following function can be added:\\
	\texttt{loadPermissionsDatabase(connection, [type], names)}\\
	This function loads the permission and inheritance rules from two tables, via \texttt{connection}, specified in \texttt{names}. The argument \texttt{type} is optional to implement. If necessary, it can carry the database type. The function does not return a value. The types of the arguments are:\\
	\begin{tabular}{l|p{.7\textwidth}}
		\textbf{Name} & \textbf{Type}\\\hline
		\texttt{connection} & database connection\\\hline
		\texttt{type} & enum\\\hline
		\texttt{names} & a key-value object containing the names of the permission and inheritance table and the names of the relevant columns in those tables, i.e. the columns containing the names of the inheriting role, inherited role for the inheritance rules and the allowance boolean (allow/deny), role name and path expression for the permission rules
	\end{tabular}

\end{document}
