\documentclass{article}

\usepackage{a4wide}
\usepackage{algpseudocode}
\usepackage{algorithm}
\usepackage{float}
\usepackage{datetime2}

\title{Algorithms to be used in the Talos implementations}

\begin{document}
	\maketitle

	\section{Building the permission tree}
	The permission tree of a user consists of nodes. Each node represents a resource or group of resources. The process of building the tree consists of repeatedly adding a rule to the (at first) empty tree.\\
	When adding rule, traverse or create the tree, repeatedly taking the node corresponding to the first part of the rule's path and removing that part from the path. Ultimately, when the path is empty, set the permission of the resulting node to the rule's permission.
	
	\section{Adding heritage to the permission tree}
	Alongside the permissions that are specified by permission rules for a user, a user can inherit permissions from his ancestors. These permissions can be added to the permission tree. This is achieved by merging the permission tree of the user's parent (the one with his inherited permissions included) into the permission tree of the user. This merging happens in the obvious way, but child nodes that come from different users are kept separated. Also, when the permission of a node in the user's tree has been set, no nodes are added to the subtree of this node.

	\section{Nodes of the permission tree}
	A node of the permission tree can be of different types, each having its own appearance in a permission rule:\\
	\begin{tabular}{l|c|l}
		\textbf{Name} & \textbf{Notation} & \textbf{Meaning} \\\hline
		Literal node & \texttt{name} & Matching exactly the specified name. \\\hline
		Variable node & \texttt{[variable]} & Matching the value of the specified variable. \\\hline
		Set node & \texttt{\{set\}} & Matching any value in the specified set. \\\hline
		Universal node & \texttt{*} & Matching any value
	\end{tabular}

	\section{Resolving a user's permission to access a resource}
	\subsection{Prerequisites}
	When a user's permission is checked, the following information is given:
	\begin{itemize}
		\item The path of the resource permission is being checked for.
		\item The permission trees of the user and his ancestors (the users he inherits from).
		\item A set of "variables": key-value pairs, the values being strings.
		\item A set of "sets": key-value pairs, the values being string arrays.
	\end{itemize}

	\subsection{The algorithm}
	Start at the root of the permission tree. For the nth node you come across, consider any child node that matches the nth part of the path, using the following prioritization:
	\begin{enumerate}
		\item The degree of ancestry the child node is inherited from. The closer the ancestor is to the user, the higher the priority.
		\item Using the following order: literal node, variable node, set node, universal node. That is, from most specific to least specific.
		\item The order of specification in the permission rules.
	\end{enumerate}
	If the path is empty, or if there are no more child nodes, check if the permission of the node, or of any of its ancestors has been set. If so, return the permission of the ancestor node closest to the node under consideration. If not, go one level up and check the next matched node there.\\
	If no permission is found in the entire tree, permission is denied for the specified.
\end{document}
